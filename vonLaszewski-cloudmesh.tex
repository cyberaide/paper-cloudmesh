\documentclass{tex/sig-alternate} 

\newcommand{\TITLE}{Federating Clouds with Cloudmesh}
\newcommand{\AUTHOR}{Gregor von Laszewski} 

%%%%%%%%%%%%%%%%%%%%%%%%%%%%%%%%%%%%%%%%%%%%%%%%%%%%%%%%%%%%%%
% LATEX DEFINITIONS 
%%%%%%%%%%%%%%%%%%%%%%%%%%%%%%%%%%%%%%%%%%%%%%%%%%%%%%%%%%%%%%

\usepackage{textcomp}
\usepackage{mathabx}
\usepackage{enumerate}
\usepackage{hyperref} 
\usepackage{array} 
\usepackage{graphicx} 
\usepackage{booktabs} 
\usepackage{pifont} 
\usepackage{todonotes} 
\usepackage{rotating} 
\usepackage{color} 
\usepackage{tabularx} 
\usepackage{amssymb}
 
\newcommand*\rot{\rotatebox{90}} 
 
\newcommand{\FILE}[1]{\todo[color=green!40]{#1}} 
 
%%%%%%%%%%%%%%%%%%%%%%%%%%%%%%%%%%%%%%%%%%%%%%%%%%%%%%%%%%%%%%
% HYPERSETUP 
%%%%%%%%%%%%%%%%%%%%%%%%%%%%%%`%%%%%%%%%%%%%%%%%%%%%%%%%%%%%%%%

\hypersetup{ 
    bookmarks=true,         % show bookmarks bar 
    unicode=false,          % non-Latin characters in AcrobatΓÇÖs bookmarks 
    pdftoolbar=true,        % show AcrobatΓÇÖs toolbar 
    pdfmenubar=true,        % show AcrobatΓÇÖs menu 
    pdffitwindow=false,     % window fit to page when opened 
    pdfstartview={FitH},    % fits the width of the page to the window 
    pdftitle={\TITLE},    % title 
    pdfauthor={\AUTHOR},     % author 
    pdfsubject={Subject},   % subject of the document 
    pdfcreator={\AUTHOR},   % creator of the document 
    pdfproducer={\AUTHOR}, % producer of the document 
    pdfkeywords={hindex} {cloud}{FutureGrid}, % list of keywords 
    pdfnewwindow=true,      % links in new window 
    colorlinks=false,       % false: boxed links; true: colored links 
    linkcolor=red,          % color of internal links (change box color with linkbordercolor) 
    citecolor=green,        % color of links to bibliography 
    filecolor=magenta,      % color of file links 
    urlcolor=cyan           % color of external links 
} 
 
%%%%%%%%%%%%%%%%%%%%%%%%%%%%%%%%%%%%%%%%%%%%%%%%%%%%%%%%%%%%%%%%%%%%%% 
% HYPHENATION
%%%%%%%%%%%%%%%%%%%%%%%%%%%%%%%%%%%%%%%%%%%%%%%%%%%%%%%%%%%%%%%%%%%%%% 

\hyphenation{cloud-mesh}

%%%%%%%%%%%%%%%%%%%%%%%%%%%%%%%%%%%%%%%%%%%%%%%%%%%%%%%%%%%%%%%%%%%%%% 

\begin{document} 
% 
% --- Author Metadata here --- 
\conferenceinfo{TBD}{} 
\CopyrightYear{2014}  
\crdata{X-XXXXX-XX-X/XX/XX}  % Allows default copyright data (0-89791-88-6/97/05) to be over-ridden - IF NEED BE. 
% --- End of Author Metadata --- 
 
\title{\TITLE} 
%\subtitle{[Extended Abstract] 
%\titlenote{A full version of this paper is available as 
%\texttt{www.acm.org/eaddress.htm}}} 
 
\numberofauthors{1}  
\author{ 
\alignauthor 
Gregor von Laszewski,\titlenote{Corresponding Author.} 
Fugang Wang, 
Geoffrey C. Fox, 
Others to be added ...  \\
       \affaddr{Indiana University}\\
       \affaddr{2719 10th Street}\\
       \affaddr{ Bloomington, Indiana, U.S.A.}\\ 
       \email{laszewski@gmail.com} 
} 
\date{13 March 2014} 
 
\toappear{} 
\maketitle 
\begin{abstract} 

This paper analyses different aspects of federating clouds and outline
a simple design that can practically used by users to create their own
cloud federation based on managing their own credentials either in a
hosted service, or in a service that is hosted by the
user. Furthermore we are discussing implications and solutions for a
federated unified metrics system. Lastly we will discuss the setup of
a federated cloud managed by administrators based on a mutual trust
model. We have developed a number of components that support the
creation of such a federated environment. This system is known as
cloudmesh and has been used in practice to achieve virtual machine
federations. An important distinguishing factor of cloudmesh is that
it also allows the use of bare metal provisioning for authorized
users, offering services beyond those available by typical
clouds. 

 
\end{abstract} 
 
% A category with the (minimum) three required fields 
\category{H.4}{Information Systems Applications}{Miscellaneous} 
%A category including the fourth, optional field follows... 
\category{D.2.8}{Software Engineering}{Metrics}[complexity measures, performance measures] 
 
\terms{Theory} 
 
\keywords{Cloud}
 
%%%%%%%%%%%%%%%%%%%%%%%%%%%%%%%%%%%%%%%%%%%%%%%%%%%%%%%%%%%%%%%%%%%%%% 
% SECTIONS 
%%%%%%%%%%%%%%%%%%%%%%%%%%%%%%%%%%%%%%%%%%%%%%%%%%%%%%%%%%%%%%%%%%%%%% 

%%%%%%%%%%%%%%%%%%%%%%%%%%%%%%%%%%%%%%%%%%%%%%%%%%%%%%%%%%%%%%%%%%%%%%
\section{Introduction}
%%%%%%%%%%%%%%%%%%%%%%%%%%%%%%%%%%%%%%%%%%%%%%%%%%%%%%%%%%%%%%%%%%%%%%

\subsection{FutureGrid}



FutureGrid \cite{las2010gce,las12fg-bookchapter} ``is a project led by
Indiana University and funded by the National Science Foundation (NSF)
to develop a high performance grid test bed that will allow scientists
to collaboratively develop and test innovative approaches to parallel,
grid, and cloud computing. FutureGrid will provide the infrastructure
to researchers that allows them to perform their own computational
experiments using distributed systems. The goal is to make it easier
for scientists to conduct such experiments in a transparent manner.
FutureGrid users will be able to deploy their own hardware and
software configurations on a public/private cloud, and run their
experiments. They will be able to save their configurations and
execute their experiments using the provided tools. The FutureGrid
test bed is composed of a high speed network connecting distributed
clusters of high performance computers. FutureGrid employs
virtualization technology that will allow the test bed to support a
wide range of operating systems.''



FutureGrid contains a number compute resources organized as part of
clusters with different types and size. They are interconnected with up to a 10GB Ethernet among its sites. The sites include Indiana University, University of Chicago, San Diego Supercomputing Center, Texas Advanced Computing Center, and University of Florida.
In total 481 compute servers with 1126 CPUs and 4496 Cores are
offered. In addition it offers also 448 GPU cores. The total RAM is
about 21.5TB. Secondary storage is about 1PB. A more detailed
description of the resources is provided in
\cite{vonLaszewski-bigdata-bookchapter2014}

FutureGrid offers a very rich environment to its users. We distinguish the following categories: Cloud PaaS, IaaS, GridaaS, HPCaaS, TestbedaaS.

FutureGrid provides an advanced framework to manage user and project affiliation and propagates this information to a variety of subsystems constituting the FG service infrastructure. This includes operational services to deal with authentication, authorization and accounting. In particular we have developed a unique metric framework that allows us to create usage reports from our entire Infrastructure as a Service frameworks. Repeatable experiments can be created with a number of tools including Pegasus, Precip and Cloudmesh.Provisioning of services and images can be conducted by RAIN \cite{imagemanagement,fg-1295}. Infrastructure monitoring is enabled via Nagios \cite{nagios}, Ganglia \cite{ganglia}, and Inca \cite{inca} and our own cloud metric system \cite{las08federated-cloud}.
Within the traditional high performance computing services FG offers a traditional MPI/batch queuing system and a virtual large memory system that are beneficial for big data calculations.


One of the main features of FutureGrid is to offer its users a variety
of infrastructure as a service frameworks
\cite{comparisoncloud,las2011virt} including OpenStack, Eucalyptus,
and Nimbus. These frameworks provide Based on our experience
with FutureGrid over the last couple of years, it is advantageous to
offer a mixed operation model. This includes a standard production
cloud that operates on-demand, but also a set of cloud instances that
can be reserved for a particular project. We have conducted this for
several projects in FutureGrid, including those that required
dedicated access to resources as part of big data research such as
classes \cite{fg405,fg368} or research projects with extremely large
virtual machines \cite{fg298}.


\subsection{Federation}

Cloud federation

Account based federation

Administrative federation

\url{http://www.egi.eu/infrastructure/cloud/}


\cite{kurze2011cloudfederation}


\subsection{Definitions}

To define the federated services that we support with our efforts, we will be
using the following definitions throughout the paper.

\begin{description}
   \setlength{\itemsep}{0pt}
   \setlength{\parsep}{0pt}

\item[public-cloud:]  a service provider makes resources available to
  usres over the public Internet. This includes compute, storage, and
  applications. FutureGrid offers to its users a public cloud. 

\item[private-cloud:] access to services may be having additional
  restrictions. Restrictions could include a limited set of authorized
  users to the services offered or  possible restrictions of exposing services on the public
  internet. FutureGrid offers the ability to set up private clouds for
  special projects. Examples include modified OpenStack deployments or
  reserved resources for classes.

\item[hybrid-cloud:] a combination of public and private clouds. 

\item[multi-cloud:] access to a number of different clouds that may
  even use different IaaS or PaaS offerings. 

\item[hpc-service:] a cloud service that allows the ability to run high
  performance computing jobs, for example on a compute cluster
  offering MPI. 

\item[provisioning:] a set of instructions to install the operating
  system, data and software to enable access to it. 

\item[rain:] an advanced set of instructions that not only provisions
  the operating system, but allows the deployment and configuration of
  useful and complex services to be run on one or multiple machines in
  order to provide a service utilizing potentially distributed
  resources or services.  It also contains the ability to re-provision
  servers and services, that is services may be suspended and the
  resources used to run the service may be used by other services.

\end{description}

%%%%%%%%%%%%%%%%%%%%%%%%%%%%%%%%%%%%%%%%%%%%%%%%%%%%%%%%%%%%%%%%%%%%%%
\section{Related Work}
%%%%%%%%%%%%%%%%%%%%%%%%%%%%%%%%%%%%%%%%%%%%%%%%%%%%%%%%%%%%%%%%%%%%%%

\begin{description}

\item[Phantom] \cite{phantom12,www-phantom} is a tools that monitors the health of resources and
  automatically provisions and configures new ones based on demand. IT
  is designed to automatically (a) provisioning new
  resources to counteract failures or (b) react to increasing demand
  reducing  constant attention and repetitive task that can be
  automated. 

\item[RightScale] \cite{Rightscale} enables users to manage multi cloud infrastructure
  by migrating workloads between private clouds and public
  clouds. RightScale is by now interfacing with a wide variety of
  clouds including Amazon Web Services (AWS), Rackspace Cloud,
  Windows Azure, and Google Compute Engine. In addition it also offers
  a cloud cost estimator allowing customers to assess expenses 
  they are charged by comparing their workload on various cloud
  providers.

\item[StarCluster] \cite{www-starcluster} ``StarCluster is an open source cluster-computing toolkit for Amazon's Elastic Compute Cloud (EC2). StarCluster has been designed to simplify the process of building, configuring, and managing clusters of virtual machines on Amazon's EC2 cloud.''

\end{description}




spot pricing, mit 

\begin{verbatim}
http://www.aifb.kit.edu/images/0/02/Cloud_Federation.pdf
\end{verbatim}

library federation

protocol

intercloud

hybrid cloud federation

regions


Icehouse
Federation is the first new feature listed on Keystone's release notes for Icehouse:

  \url{https://wiki.openstack.org/wiki/ReleaseNotes/Icehouse#OpenStack_Identity_.28Keystone.29}



%%%%%%%%%%%%%%%%%%%%%%%%%%%%%%%%%%%%%%%%%%%%%%%%%%%%%%%%%%%%%%%%%%%%%%
\section{Requirements}
%%%%%%%%%%%%%%%%%%%%%%%%%%%%%%%%%%%%%%%%%%%%%%%%%%%%%%%%%%%%%%%%%%%%%%

TBD

%%%%%%%%%%%%%%%%%%%%%%%%%%%%%%%%%%%%%%%%%%%%%%%%%%%%%%%%%%%%%%%%%%%%%%
\section{Design}
%%%%%%%%%%%%%%%%%%%%%%%%%%%%%%%%%%%%%%%%%%%%%%%%%%%%%%%%%%%%%%%%%%%%%%

TBD
%%%%%%%%%%%%%%%%%%%%%%%%%%%%%%%%%%%%%%%%%%%%%%%%%%%%%%%%%%%%%%%%%%%%%%
\section{Design}
%%%%%%%%%%%%%%%%%%%%%%%%%%%%%%%%%%%%%%%%%%%%%%%%%%%%%%%%%%%%%%%%%%%%%%

TBD

%%%%%%%%%%%%%%%%%%%%%%%%%%%%%%%%%%%%%%%%%%%%%%%%%%%%%%%%%%%%%%%%%%%%%%
\section{Implementation}
%%%%%%%%%%%%%%%%%%%%%%%%%%%%%%%%%%%%%%%%%%%%%%%%%%%%%%%%%%%%%%%%%%%%%%

TBD

%%%%%%%%%%%%%%%%%%%%%%%%%%%%%%%%%%%%%%%%%%%%%%%%%%%%%%%%%%%%%%%%%%%%%%
\section{Status and Future Work}
%%%%%%%%%%%%%%%%%%%%%%%%%%%%%%%%%%%%%%%%%%%%%%%%%%%%%%%%%%%%%%%%%%%%%%

TBD

%%%%%%%%%%%%%%%%%%%%%%%%%%%%%%%%%%%%%%%%%%%%%%%%%%%%%%%%%%%%%%%%%%%%%%
\section{Conclusion}
%%%%%%%%%%%%%%%%%%%%%%%%%%%%%%%%%%%%%%%%%%%%%%%%%%%%%%%%%%%%%%%%%%%%%%

TBD
 
%%%%%%%%%%%%%%%%%%%%%%%%%%%%%%%%%%%%%%%%%%%%%%%%%%%%%%%%%%%%%%%%%%%%%% 
% Acknowledgment 
%%%%%%%%%%%%%%%%%%%%%%%%%%%%%%%%%%%%%%%%%%%%%%%%%%%%%%%%%%%%%%%%%%%%%% 
 
\section*{Acknowledgement} 
 
This work is part of the Technology Auditing Service (TAS) project sponsored by NSF under grant number OCI 1025159. Lessons learned from FutureGrid have significantly influenced this work. 

\section{OLD}


\section{Cloudmesh}\label{S:cloudmesh}


At \cite{github-cloudmesh} we find an extensive set of information about Cloudmesh that is cited within this section. 
From the experience with FutureGrid we identified the need for a more tightly integrated software infrastructure addressing the need to deliver a software-defined system encompassing virtualized and bare-metal infrastructure, networks, application, systems and platform software with a unifying goal of providing Cloud Testbeds as a Service (CTaaS). This system is termed Cloudmesh to symbolize 


\begin{enumerate}[(a)]


\item the creation of a tightly integrated mesh of services targeting multiple IaaS frameworks 


\item the ability to federate a number of resources from academia and industry. This includes existing FutureGrid infrastructure, Amazon Web Services, Azure, HP Cloud, Karlsruhe using not only one IaaS framework but various. 


\item the creation of an environment in which it becomes more easy to experiment with platforms and software services while assisting to deploy them more easily.  


\end{enumerate}


In addition to virtual resources, FutureGrid exposes bare-metal provisioning to users, but also a subset of HPC monitoring infrastructure tools. Services will be available through command line, API, and Web interfaces.


\subsection{Functionality}


Due to its integrated services Cloudmesh provides the ability to be an onramp for other clouds. It also provides information services to various system level sensors to give access to sensor and utilization data. They internally can be used to optimize the system usage. The provisioning experience from FutureGrid has taught us that we need to provide the creation of new clouds, the repartitioning of resources between services (cloud shifting), and the integration of external cloud resources in case of over provisioning (cloud bursting). As we deal with many IaaS we need an abstraction layer on top of the IaaS framework. Experiment management is conducted with workflows controlled in shells \cite{cmd3}, Python/iPython, as well as systems such as OpenStack?s Heat, Accounting is supported through additional services such as user management and charge rate management. Not all features are yet implemented. Figure \label{F:cm-func} shows the main functionality that we target at this time to implement.


\begin{figure}[htb]
  \centering
    \includegraphics[width=1.0\columnwidth]{images/cm-functionality.pdf}
  \caption{CM Functionality.}\label{F:cm-func}
\end{figure}




\subsection{Architecture}


The three layers of the Cloudmesh architecture include a Cloudmesh Management Framework for monitoring and operations, user and project management, experiment planning and deployment of services needed by an experiment, provisioning and execution environments to be deployed on resources to (or interfaced with) enable experiment management, and resources.


\begin{figure}[htb]
  \centering
    \includegraphics[width=1.0\columnwidth]{images/cm-arch.pdf}
  \caption{CM Architecture.}
\end{figure}


\paragraph{System Monitoring and Operations.}


The management framework contains services to facilitate FutureGrid day-to-day operation, including federated or selective monitoring of the infrastructure. Cloudmesh leverages FutureGrid for the operational services and allows administrators to view ongoing system status and experiments, as well as interact with users through ticket systems and messaging queues to inform subscribed users on the status of the system.
The Cloudmesh management framework offers services that simplify integration of resources in the FutureGrid nucleus or through federation. This includes, for user management, access to predefined setup templates for services in enabling resource and service provisioning as well as experiment execution. To integrate IaaS frameworks Cloudmesh offers two distinct services:


(a) a federated IaaS frameworks hosted on FutureGrid,
(b) the availability of a service that is hosted on FutureGrid, allowing the integration of IaaS frameworks through user credentials either registered by the users or automatically obtained from our distributed user directory.


For (b) several toolkits exist to create user-based federations, including our own abstraction level which supports interoperability via libcloud, but more importantly it supports directly the native OpenStack protocol and overcomes limitations of the EC2 protocol and the libcloud compatibility layer. Plugins that we currently develop will enable access to clouds via firewall penetration, abstraction layers for clouds with few public IP addresses and integration with new services such as OpenStack Heat. We successfully federated resources from Azure, AWS, the HP cloud, Karlsruhe Institute of Technology Cloud, and four FutureGrid clouds using various versions of OpenStack and Eucalyptus. The same will be done for OpenCirrus resources at GT and CMU through firewalls or proxy servers.
Additional management flexibility will be introduced through automatic cloud-bursting and shifting services. While cloud bursting will locate empty resources in other clouds, cloud shifting will identify unused services and resources, shut them down and provision them with services that are requested by the users. We have demonstrated this concept in 2012 moving resources for more than 100 users to services that were needed based on class schedules. A reservation system will be used to allow for reserved creation of such environments, along with improvements of automation of cloud-shifting.


\paragraph{User and Project Services}


FutureGrid user and project services simplify the application processes needed to obtain user accounts and projects. We have demonstrated in FutureGrid the ability to create accounts in a very short time, including vetting projects and users – allowing fast turn-around times for the majority of FutureGrid projects with an initial startup allocation. Cloudmesh re-uses this infrastructure and also allows users to manage proxy accounts to federate to other IaaS services to provide an easy interface to integrate them.


\paragraph{Accounting and App Store}


To lower the barrier of entry Cloudmesh will be providing a shopping cart which will allow checking out of predefined repeatable experiment templates. A cost is associated with an experiment making it possible to engage in careful planning and to save time by reusing previous experiments. Additionally, the Cloudmesh App Store may function as a clearing-house of images, image templates, services offered and provisioning templates. Users may package complex deployment descriptions in an easy parameter/form-based interface and other users may be able to replicate the specified setup with.
Due to our advanced Cloudmesh Metrics framework we are in the position to further develop an integrated accounting framework allowing a usage cost model for users and management to identify the real impact of an experiment on resources. This will be useful to avoid overprovisioning and inefficient resource usage. The cost model will be based not only on number of core hours used, but also the capabilities of the resource, the time, and special support it takes to set up the experiment. We will expand upon the metrics framework of FutureGrid that allows measuring of VM and HPC usage and associate this with cost models. Benchmarks will be used to normalize the charge models.


\paragraph{Networking}


in{ceWe have a broad vision of resource integration in FutureGridources be with systems offering different levels of control from bare metal to use of a portion of a resource. Likewise, we must utilize networks offering various levels of control, from standard IP connectivity to completely configurable SDNs as novel cloud architectures will almost certainly leverage NaaS and SDN alongside system software and middleware. FutureGrid resources will make use of SDN using OpenFlow whenever possible and the same level of networking control will not be available in every location.


\paragraph{Monitoring}


To serve the purposes of CISE researchers, Cloudmesh must be able to access empirical data about the properties and performance of the underlying infrastructure beyond what is available from commercial cloud environments. To accommodate this requirement we have developed a uniform access interface to virtual machine monitoring information available for OpenStack, Eucalyptus, and Nimbus. In the future, we will be enhancing the access to historical user information. Right now they are exposed through predefined reports that we create on a regular basis. To achieve this we will also leverage the ongoing work while using the AMPQ protocol. Furthermore, Cloudmesh will provide access to common monitoring infrastructure as provided by Ganglia, Nagios, Inca, perfSonar, PAPI and others.

\subsection{Cloud Shifting}


We have already demonstrated via the RAIN tool in Cloudmesh that it is possible to easily shift resources between services. We are currently expanding upon this idea and developing more easy to use user interfaces that assist administrators and users through role and project based authentication to move resources from one service to another (see Figure \ref{F:shift}).


\begin{figure}[htb]
  \centering
    \includegraphics[width=1.0\columnwidth]{images/shift2.pdf}
  \caption{Shifting resources makes it possible to offer flexibility
    in the service distribution in case of over or underprovisioning.}\label{F:shift}
\end{figure}
v

\subsection{Graphical User Interface}


Despite the fact that Cloudmesh was originally a quite sophisticated command shell and command line tool, we have spend recently more time in exposing this functionality through a convenient Web interface. Some more popular views if this interface are depicted in Figure \ref{F:instances} hinting on how easy it is with a single button to create multiple VMs across a variety of IaaS. Also nice is that this not only includes resources at IU but also at external locations. Pushing this easy management in a more sophisticated experience for the user while associating one-click deployments that include the ability to deploy virtual clusters, Hadoop environments, and other more elaborate setups we provide an early prototype screenshot in Figure \ref{F:oneclick}.


\begin{figure}[htb]
  \centering
    \includegraphics[width=1.0\columnwidth]{images/rainbow.pdf}
  \caption{Monitoring the Service distribution of FutureGrid with Cloudmesh.}
\end{figure}


\begin{figure}[htb]
  \centering
    \includegraphics[width=1.0\columnwidth]{images/instances.pdf}
  \caption{Screenshot demonstrating how easy ot is to manage multible VMs accross various clouds.}\label{F:instances}
  \centering
    \includegraphics[width=1.0\columnwidth]{images/oneclick.pdf}
  \caption{One click deployment of platforms and sophisticated
    services that could even spawn multiple resources.}\label{F:oneclick}
\end{figure}

\section{Cloud Metrics}

%\begin{figure}[htb]
%  \centering
%    \includegraphics[width=1.0\columnwidth]{images/cloudmesh-metrics.pdf}
%  \caption{Metric Architecture}
%  \label{F:metric-arch}
%\end{figure}

Figure 1. System design of CloudMesh Metrics

\subsection{CloudMesh Metrics: Measuring Resource Utilization}

In our system of FutureGrid, summary of resource allocation data is important to provide the visibility of resource utilization for whole clusters in the system. Over 300 projects and almost 3000 users in FutureGrid have accounts in several resources with various service partitions. Merging the resource allocation data into a single place meets the challenge in this environment. FutureGrid offers 5 service partitions i.e. Torque, Moab, Nimbus, Eucalyptus, and OpenStack in 8 clusters such as India, Sierra, etc with 5 different geological regions. With the variety of services and platforms, the measurement of the resource allocation is needed to be merged in order to provide overview of resource usage for their projects and accounts. The two main components of CloudMesh Metrics in Figure 1 measure the resource allocation across several IaaS platforms and provide statistics and usage data through various user interface. Metric Collector starts with OpenStack, Eucalyptus, and Nimbus but not limited to these platforms. The pre-defined filters help collect usage data from any cloud platforms and data resource (e.g. log files, mysql, or sqlite). Representing measured data is one of the key roles in CloudMesh Metrics to help understand the data. It is often to happen to interpret data improperly which may result in an incorrect conclusion. With visualization tools and Command-Line Interface provided by Cloud Analyzer, the chances of misuse of statistics can be diminished. External visualization toolkits and exporting tools have been used to describe usage information without unnecessary interpretation. The measured data by Metrics is represented in CloudMesh which is a universal framework of managing any compute resources. CloudMesh consists of several components that manage accounts and handle compute resources not only in FutureGrid, but also in any other compute resources. With the Rest API, the metrics data can be presented by CloudMesh.

\subsection{Metric Collector}

The process of measuring resource allocation begins with collecting usage data within the infrastructure. Each IaaS cloud resource provider typically has an internal system to record resource allocation per users or groups. Virtual Instance is a good measure of resource utilization in cloud environments since it reserves and releases compute resource when it deploys. To measure the amount of resources allocated and consumed, CloudMesh Metrics places the Metric Collector to IaaS platforms. For example, the metric collector periodically connects to Openstack database to obtain information of VM deployment. Once the collecting process has been completed, unifying different data types of the measured data is critical to provide a summary of resource allocation across several IaaS platforms. The filter of the metric collector converts the data types into a single data object consisting of attribute-value pairs in Metrics database.To achieve this goal, NoSQL database is recommended in the metrics system.  Table 1 shows  details of measuring resource allocation on FutureGrid.

Note:
Table \ref{T:compare-iaas}

    Nimbus create\_event, remove\_event tables 
    OS: nova, keystone  databases 

    Euca: cc.log, Compute Log 


%\newcommand{\YES}{\checkmark}
%\newcommand{\NO}{\textopenbullet}
\newcommand{\YES}{\ding{51}}

\newcommand{\NO}{\ding{55}}


\begin{table}[h!]
  \caption{Measurement of IaaS on FutureGrid}\label{T:compare-iaas}
  ~\\
  \begin{small}
  \begin{tabularx}{\columnwidth}{|l|X|X|X|}
  \hline
                 & {\bf Nimbus} & {\bf OpenStack} & {\bf Eucalyptus} \\
    \hline
    \hline
    \multicolumn{4}{|l|}{\bf Documentation of the Data Sources} \\
    \hline
       & \NO & \YES & \YES \\
    \hline
    \hline
    \multicolumn{4}{|l|}{\bf Data Sources} \\
    \hline
         & sqlite3 & MySQL & Log Files \\
    \hline
    \hline
    \multicolumn{4}{|l|}{\bf Metrics} \\
    \hline
    ~~vCPU core & \YES & \YES & \YES \\
    ~~memory & \YES & \YES & \YES \\
    ~~disk & \YES & \YES & \YES \\
    ~~instance type   & \NO? & \YES & \NO? \\
    ~~host & \NO? & \YES & \NO? \\
    \hline
    \hline
    \multicolumn{4}{|l|}{\bf Account Management Features} \\
    \hline
    ~~Users     & \YES & \YES & \YES \\
    ~~Projects & \NO & \YES & \YES \\
    \hline
    \hline
    \multicolumn{4}{|l|}{\bf Cluster} \\
    \hline
    ~~Alamo  & \YES & \YES & \YES \\
    ~~Foxtrot & \YES & \NO & \NO \\
    ~~Hotel    & \YES & \YES & \NO \\
    ~~India     & \NO  & \YES & \YES? \\
    ~~Sierra    & \NO & \YES & \YES? \\
    ~~Lima     & ?       &  ?      &  ?       \\   
    \hline
%    Region& \shortstack[l]{TACC$^1$, \\UF$^2$, \\UChicago$^3$, \\SDSC$^4$} & \shortstack[l]{TACC, \\IU$^5$, \\SDSC } & \shortstack[l]{IU, \\SDSC$^6$} \\
%    \hline
  \end{tabularx}\\
%  $^1$ Texas Advanced Computing Center \\
%  $^2$ University of Florida \\
%  $^3$ University of Chicago \\
%  $^4$ San Diego Supercomputing Center\\
%  $^5$ Indiana University\\
%  $^6$ in early 2014\\
\end{small}
\end{table}


%\clearpage 
 
%\bibliographystyle{IEEEtranS}
\bibliographystyle{IEEEtran}
%\bibliographystyle{abbrv} 
\bibliography{% 
bib/vonLaszewski-jabref.bib,%
bib/cyberaide-cloud} 
 
\end{document} 



\begin{table*}
\caption{Table}
\begin{tabular}{lllllll}
IaaS &
Resource Data Location &
Metric &
Account Info &
Data Access Point &
Cluster &
Region\\
\hline
Nimbus &
sqlite3 &
vCPU core, memory, disk &
User only &
create\_event, remove\_event tables &
Alamo, Sierra, Hotel, Foxtrot &
Texas Advanced Computing Center, University of Florida, University of
Chicago, San Diego Supercomputing Center\\
OpenStack &
MySQL &
vCPU core, memory, disk, instance type, physical location &
User and Project &
nova, keystone databases &
India, Sierra, Alamo &
Texas Advanced Computing Center, Indiana University, San Diego
Supercomputing Center \\
Eucalyptus &
Logs &
vCPU core, memory, disk &
User only &
cc.log (Compute Cloud) &
India &
Indiana University, San Diego Supercomputing Center* (in early 2014) \\
\hline
\end{tabular}
\end{table*}
