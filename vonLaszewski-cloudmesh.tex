\documentclass{tex/sig-alternate} 

\newcommand{\TITLE}{Accessing Multiple  Clouds with Cloudmesh}
\newcommand{\AUTHOR}{Gregor von Laszewski} 

%%%%%%%%%%%%%%%%%%%%%%%%%%%%%%%%%%%%%%%%%%%%%%%%%%%%%%%%%%%%%%
% LATEX DEFINITIONS 
%%%%%%%%%%%%%%%%%%%%%%%%%%%%%%%%%%%%%%%%%%%%%%%%%%%%%%%%%%%%%%

\usepackage[T1]{fontenc}
\usepackage{adjustbox}
\usepackage[usenames,dvipsnames]{color}
\usepackage{color}
\usepackage{textcomp}
\usepackage{mathabx}
\usepackage{enumerate}
\usepackage{hyperref} 
\usepackage{array} 
\usepackage{graphicx} 
\usepackage{booktabs} 
\usepackage{pifont} 
\usepackage{todonotes} 
\usepackage{rotating} 
\usepackage{color} 
\usepackage{tabularx} 
\usepackage{amssymb}
 
\newcommand*\rot{\rotatebox{90}} 
 
\newcommand{\FILE}[1]{\todo[color=green!40]{#1}} 
 
%%%%%%%%%%%%%%%%%%%%%%%%%%%%%%%%%%%%%%%%%%%%%%%%%%%%%%%%%%%%%%
% HYPERSETUP 
%%%%%%%%%%%%%%%%%%%%%%%%%%%%%%`%%%%%%%%%%%%%%%%%%%%%%%%%%%%%%%%

\hypersetup{ 
    bookmarks=true,         % show bookmarks bar 
    unicode=false,          % non-Latin characters in AcrobatΓÇÖs bookmarks 
    pdftoolbar=true,        % show Acrobat's toolbar 
    pdfmenubar=true,        % show Acrobat's menu 
    pdffitwindow=false,     % window fit to page when opened 
    pdfstartview={FitH},    % fits the width of the page to the window 
    pdftitle={\TITLE},    % title 
    pdfauthor={\AUTHOR},     % author 
    pdfsubject={Subject},   % subject of the document 
    pdfcreator={\AUTHOR},   % creator of the document 
    pdfproducer={\AUTHOR}, % producer of the document 
    pdfkeywords={hindex} {cloud}{FutureGrid}, % list of keywords 
    pdfnewwindow=true,      % links in new window 
    colorlinks=false,       % false: boxed links; true: colored links 
    linkcolor=red,          % color of internal links (change box color with linkbordercolor) 
    citecolor=green,        % color of links to bibliography 
    filecolor=magenta,      % color of file links 
    urlcolor=cyan           % color of external links 
} 
 
%%%%%%%%%%%%%%%%%%%%%%%%%%%%%%%%%%%%%%%%%%%%%%%%%%%%%%%%%%%%%%%%%%%%%% 
% HYPHENATION
%%%%%%%%%%%%%%%%%%%%%%%%%%%%%%%%%%%%%%%%%%%%%%%%%%%%%%%%%%%%%%%%%%%%%% 

\hyphenation{cloud-mesh}

%%%%%%%%%%%%%%%%%%%%%%%%%%%%%%%%%%%%%%%%%%%%%%%%%%%%%%%%%%%%%%%%%%%%%% 

\begin{document} 
% 
% --- Author Metadata here --- 
\conferenceinfo{TBD}{} 
\CopyrightYear{2014}  
\crdata{X-XXXXX-XX-X/XX/XX}  % Allows default copyright data (0-89791-88-6/97/05) to be over-ridden - IF NEED BE. 
% --- End of Author Metadata --- 
 
\title{\TITLE} 
%\subtitle{[Extended Abstract] 
%\titlenote{A full version of this paper is available as 
%\texttt{www.acm.org/eaddress.htm}}} 
 
\numberofauthors{1}  
\author{ 
\alignauthor 
Gregor von Laszewski,\titlenote{Corresponding Author.} 
Fugang Wang, 
%Hyungro Lee,
Geoffrey C. Fox, \\
       \affaddr{School of Informatics and Computing, Indiana University}\\
       \affaddr{919 E. 10th Street}\\
       \affaddr{Bloomington IN 47408, U.S.A.}\\ 
       \email{laszewski@gmail.com} 
} 
\date{13 March 2014} 
 
\toappear{} 
\maketitle 
\begin{abstract} 

  This paper analyses different aspects of managing multiple clouds
  clouds. We present the design of a toolkit design that can
  practically used by users and administrators to manage multicloud
  environments.  It can either be run by individual users or offered
  as a service to a shared user community. We have practically,
  demonstrated its use as part of a FutureGrid service allowing users
  users of FutureGrid to utilize such a service with their
  credentials. Furthermore, we are discussing implications and
  solutions for a unified metrics system assisting the users to find
  and utilize resources appropriate for their applications. Lastly we
  discuss how to move such a multicloud environment forward by
  integrating clouds managed by the community or are offered as public
  clouds. This includes the introduction of a mutual trust agreements
  on user and project baseis. We have developed a number of components
  that support the creation of such a multicloud environment. This
  system is known as Cloudmesh and has been used in practice to
  achieve virtual machine management in multiple clouds. An important
  distinguishing factor of Cloudmesh is that it also allows the use of
  bare metal provisioning for supporting service providers and
  authorized users, offering services beyond those available by
  typical clouds.

  The paper will be 8 pages long 
 
\end{abstract} 
 
% A category with the (minimum) three required fields 
\category{H.4}{Information Systems Applications}{Miscellaneous} 
%A category including the fourth, optional field follows... 
\category{D.2.8}{Software Engineering}{Metrics}[complexity measures, performance measures] 
 
\terms{Theory} 
 
\keywords{Cloud}
 
%%%%%%%%%%%%%%%%%%%%%%%%%%%%%%%%%%%%%%%%%%%%%%%%%%%%%%%%%%%%%%%%%%%%%% 
% SECTIONS 
%%%%%%%%%%%%%%%%%%%%%%%%%%%%%%%%%%%%%%%%%%%%%%%%%%%%%%%%%%%%%%%%%%%%%% 

%%%%%%%%%%%%%%%%%%%%%%%%%%%%%%%%%%%%%%%%%%%%%%%%%%%%%%%%%%%%%%%%%%%%%%
\section{Introduction}
%%%%%%%%%%%%%%%%%%%%%%%%%%%%%%%%%%%%%%%%%%%%%%%%%%%%%%%%%%%%%%%%%%%%%%

\subsection{FutureGrid}



FutureGrid \cite{las2010gce,las12fg-bookchapter} ``is a project led by
Indiana University and funded by the National Science Foundation (NSF)
to develop a high performance grid test bed that will allow scientists
to collaboratively develop and test innovative approaches to parallel,
grid, and cloud computing. FutureGrid will provide the infrastructure
to researchers that allows them to perform their own computational
experiments using distributed systems. The goal is to make it easier
for scientists to conduct such experiments in a transparent manner.
FutureGrid users will be able to deploy their own hardware and
software configurations on a public/private cloud, and run their
experiments. They will be able to save their configurations and
execute their experiments using the provided tools. The FutureGrid
test bed is composed of a high speed network connecting distributed
clusters of high performance computers. FutureGrid employs
virtualization technology that will allow the test bed to support a
wide range of operating systems.''



FutureGrid contains a number compute resources organized as part of
clusters with different types and size. They are interconnected with up to a 10GB Ethernet among its sites. The sites include Indiana University, University of Chicago, San Diego Supercomputing Center, Texas Advanced Computing Center, and University of Florida.
In total 481 compute servers with 1126 CPUs and 4496 Cores are
offered. In addition it offers also 448 GPU cores. The total RAM is
about 21.5TB. Secondary storage is about 1PB. A more detailed
description of the resources is provided in
\cite{vonLaszewski-bigdata-bookchapter2014}

FutureGrid offers a very rich environment to its users. We distinguish the following categories: Cloud PaaS, IaaS, GridaaS, HPCaaS, TestbedaaS.

FutureGrid provides an advanced framework to manage user and project affiliation and propagates this information to a variety of subsystems constituting the FG service infrastructure. This includes operational services to deal with authentication, authorization and accounting. In particular we have developed a unique metric framework that allows us to create usage reports from our entire Infrastructure as a Service frameworks. Repeatable experiments can be created with a number of tools including Pegasus, Precip and Cloudmesh.Provisioning of services and images can be conducted by RAIN \cite{imagemanagement,fg-1295}. Infrastructure monitoring is enabled via Nagios \cite{nagios}, Ganglia \cite{ganglia}, and Inca \cite{inca} and our own cloud metric system \cite{las08federated-cloud}.
Within the traditional high performance computing services FG offers a traditional MPI/batch queuing system and a virtual large memory system that are beneficial for big data calculations.


One of the main features of FutureGrid is to offer its users a variety
of infrastructure as a service frameworks
\cite{comparisoncloud,las2011virt} including OpenStack, Eucalyptus,
and Nimbus. These frameworks provide Based on our experience
with FutureGrid over the last couple of years, it is advantageous to
offer a mixed operation model. This includes a standard production
cloud that operates on-demand, but also a set of cloud instances that
can be reserved for a particular project. We have conducted this for
several projects in FutureGrid, including those that required
dedicated access to resources as part of big data research such as
classes \cite{fg405,fg368} or research projects with extremely large
virtual machines \cite{fg298}.


\subsection{Federation}

Cloud federation

Account based federation

Administrative federation

\url{http://www.egi.eu/infrastructure/cloud/}


\cite{kurze2011cloudfederation}


\subsection{Definitions}

To define the federated services that we support with our efforts, we will be
using the following definitions throughout the paper.

\begin{description}
   \setlength{\itemsep}{0pt}
   \setlength{\parsep}{0pt}

\item[public-cloud:]  a service provider makes resources available to
  usres over the public Internet. This includes compute, storage, and
  applications. FutureGrid offers to its users a public cloud. 

\item[private-cloud:] access to services may be having additional
  restrictions. Restrictions could include a limited set of authorized
  users to the services offered or  possible restrictions of exposing services on the public
  internet. FutureGrid offers the ability to set up private clouds for
  special projects. Examples include modified OpenStack deployments or
  reserved resources for classes.

\item[hybrid-cloud:] a combination of public and private clouds. 

\item[multi-cloud:] access to a number of different clouds that may
  even use different IaaS or PaaS offerings. 

\item[hpc-service:] a cloud service that allows the ability to run high
  performance computing jobs, for example on a compute cluster
  offering MPI. 

\item[provisioning:] a set of instructions to install the operating
  system, data and software to enable access to it. 

\item[rain:] an advanced set of instructions that not only provisions
  the operating system, but allows the deployment and configuration of
  useful and complex services to be run on one or multiple machines in
  order to provide a service utilizing potentially distributed
  resources or services.  It also contains the ability to re-provision
  servers and services, that is services may be suspended and the
  resources used to run the service may be used by other services.

\item[provider consortium:] is a (virtual) organization that
  integrates resources from multiple providers. We also can refer to
  such a consortium as a multi-cloud Grid.

\end{description}

%%%%%%%%%%%%%%%%%%%%%%%%%%%%%%%%%%%%%%%%%%%%%%%%%%%%%%%%%%%%%%%%%%%%%%
\section{Related Work}
%%%%%%%%%%%%%%%%%%%%%%%%%%%%%%%%%%%%%%%%%%%%%%%%%%%%%%%%%%%%%%%%%%%%%%

\begin{description}

\item[Phantom] \cite{phantom12,www-phantom} is a tools that monitors the health of resources and
  automatically provisions and configures new ones based on demand. IT
  is designed to automatically (a) provisioning new
  resources to counteract failures or (b) react to increasing demand
  reducing  constant attention and repetitive task that can be
  automated. 

\item[RightScale] \cite{Rightscale} enables users to manage multi cloud infrastructure
  by migrating workloads between private clouds and public
  clouds. RightScale is by now interfacing with a wide variety of
  clouds including Amazon Web Services (AWS), Rackspace Cloud,
  Windows Azure, and Google Compute Engine. In addition it also offers
  a cloud cost estimator allowing customers to assess expenses 
  they are charged by comparing their workload on various cloud
  providers.

\item[StarCluster] \cite{www-starcluster} ``StarCluster is an open source cluster-computing toolkit for Amazon's Elastic Compute Cloud (EC2). StarCluster has been designed to simplify the process of building, configuring, and managing clusters of virtual machines on Amazon's EC2 cloud.''

\end{description}




spot pricing, mit 

\begin{verbatim}
http://www.aifb.kit.edu/images/0/02/Cloud_Federation.pdf
\end{verbatim}

library federation

protocol

intercloud

hybrid cloud federation

regions


Icehouse
Federation is the first new feature listed on Keystone's release notes for Icehouse:

  \url{https://wiki.openstack.org/wiki/ReleaseNotes/Icehouse#OpenStack_Identity_.28Keystone.29}



%%%%%%%%%%%%%%%%%%%%%%%%%%%%%%%%%%%%%%%%%%%%%%%%%%%%%%%%%%%%%%%%%%%%%%
\section{Requirements}
%%%%%%%%%%%%%%%%%%%%%%%%%%%%%%%%%%%%%%%%%%%%%%%%%%%%%%%%%%%%%%%%%%%%%%

TBD

%%%%%%%%%%%%%%%%%%%%%%%%%%%%%%%%%%%%%%%%%%%%%%%%%%%%%%%%%%%%%%%%%%%%%%
\section{Design}
%%%%%%%%%%%%%%%%%%%%%%%%%%%%%%%%%%%%%%%%%%%%%%%%%%%%%%%%%%%%%%%%%%%%%%

TBD
%%%%%%%%%%%%%%%%%%%%%%%%%%%%%%%%%%%%%%%%%%%%%%%%%%%%%%%%%%%%%%%%%%%%%%
\section{Design}
%%%%%%%%%%%%%%%%%%%%%%%%%%%%%%%%%%%%%%%%%%%%%%%%%%%%%%%%%%%%%%%%%%%%%%

TBD

%%%%%%%%%%%%%%%%%%%%%%%%%%%%%%%%%%%%%%%%%%%%%%%%%%%%%%%%%%%%%%%%%%%%%%
\section{Implementation}
%%%%%%%%%%%%%%%%%%%%%%%%%%%%%%%%%%%%%%%%%%%%%%%%%%%%%%%%%%%%%%%%%%%%%%

TBD

%%%%%%%%%%%%%%%%%%%%%%%%%%%%%%%%%%%%%%%%%%%%%%%%%%%%%%%%%%%%%%%%%%%%%%
\section{Status and Future Work}
%%%%%%%%%%%%%%%%%%%%%%%%%%%%%%%%%%%%%%%%%%%%%%%%%%%%%%%%%%%%%%%%%%%%%%

TBD


\section{OLD}


\section{Cloudmesh}\label{S:cloudmesh}


At \cite{github-cloudmesh} we find an extensive set of information about Cloudmesh that is cited within this section. 
From the experience with FutureGrid we identified the need for a more tightly integrated software infrastructure addressing the need to deliver a software-defined system encompassing virtualized and bare-metal infrastructure, networks, application, systems and platform software with a unifying goal of providing Cloud Testbeds as a Service (CTaaS). This system is termed Cloudmesh to symbolize 


\begin{enumerate}[(a)]


\item the creation of a tightly integrated mesh of services targeting multiple IaaS frameworks 


\item the ability to federate a number of resources from academia and industry. This includes existing FutureGrid infrastructure, Amazon Web Services, Azure, HP Cloud, Karlsruhe using not only one IaaS framework but various. 


\item the creation of an environment in which it becomes more easy to experiment with platforms and software services while assisting to deploy them more easily.  


\end{enumerate}


In addition to virtual resources, FutureGrid exposes bare-metal provisioning to users, but also a subset of HPC monitoring infrastructure tools. Services will be available through command line, API, and Web interfaces.


\subsection{Functionality}


Due to its integrated services Cloudmesh provides the ability to be an onramp for other clouds. It also provides information services to various system level sensors to give access to sensor and utilization data. They internally can be used to optimize the system usage. The provisioning experience from FutureGrid has taught us that we need to provide the creation of new clouds, the repartitioning of resources between services (cloud shifting), and the integration of external cloud resources in case of over provisioning (cloud bursting). As we deal with many IaaS we need an abstraction layer on top of the IaaS framework. Experiment management is conducted with workflows controlled in shells \cite{cmd3}, Python/iPython, as well as systems such as OpenStack?s Heat, Accounting is supported through additional services such as user management and charge rate management. Not all features are yet implemented. Figure \label{F:cm-func} shows the main functionality that we target at this time to implement.


\begin{figure}[htb]
  \centering
    \includegraphics[width=1.0\columnwidth]{images/cm-functionality.pdf}
  \caption{CM Functionality.}\label{F:cm-func}
\end{figure}




\subsection{Architecture}


The three layers of the Cloudmesh architecture include a Cloudmesh Management Framework for monitoring and operations, user and project management, experiment planning and deployment of services needed by an experiment, provisioning and execution environments to be deployed on resources to (or interfaced with) enable experiment management, and resources.


\begin{figure}[htb]
  \centering
    \includegraphics[width=1.0\columnwidth]{images/cm-arch.pdf}
  \caption{CM Architecture.}
\end{figure}


\paragraph{System Monitoring and Operations.}


The management framework contains services to facilitate FutureGrid day-to-day operation, including federated or selective monitoring of the infrastructure. Cloudmesh leverages FutureGrid for the operational services and allows administrators to view ongoing system status and experiments, as well as interact with users through ticket systems and messaging queues to inform subscribed users on the status of the system.
The Cloudmesh management framework offers services that simplify integration of resources in the FutureGrid nucleus or through federation. This includes, for user management, access to predefined setup templates for services in enabling resource and service provisioning as well as experiment execution. To integrate IaaS frameworks Cloudmesh offers two distinct services:


(a) a federated IaaS frameworks hosted on FutureGrid,
(b) the availability of a service that is hosted on FutureGrid, allowing the integration of IaaS frameworks through user credentials either registered by the users or automatically obtained from our distributed user directory.


For (b) several toolkits exist to create user-based federations, including our own abstraction level which supports interoperability via libcloud, but more importantly it supports directly the native OpenStack protocol and overcomes limitations of the EC2 protocol and the libcloud compatibility layer. Plugins that we currently develop will enable access to clouds via firewall penetration, abstraction layers for clouds with few public IP addresses and integration with new services such as OpenStack Heat. We successfully federated resources from Azure, AWS, the HP cloud, Karlsruhe Institute of Technology Cloud, and four FutureGrid clouds using various versions of OpenStack and Eucalyptus. The same will be done for OpenCirrus resources at GT and CMU through firewalls or proxy servers.
Additional management flexibility will be introduced through automatic cloud-bursting and shifting services. While cloud bursting will locate empty resources in other clouds, cloud shifting will identify unused services and resources, shut them down and provision them with services that are requested by the users. We have demonstrated this concept in 2012 moving resources for more than 100 users to services that were needed based on class schedules. A reservation system will be used to allow for reserved creation of such environments, along with improvements of automation of cloud-shifting.


\paragraph{User and Project Services}


FutureGrid user and project services simplify the application processes needed to obtain user accounts and projects. We have demonstrated in FutureGrid the ability to create accounts in a very short time, including vetting projects and users -- allowing fast turn-around times for the majority of FutureGrid projects with an initial startup allocation. Cloudmesh reuses this infrastructure and also allows users to manage proxy accounts to federate to other IaaS services to provide an easy interface to integrate them.


\paragraph{Accounting and App Store}


To lower the barrier of entry Cloudmesh will be providing a shopping cart which will allow checking out of predefined repeatable experiment templates. A cost is associated with an experiment making it possible to engage in careful planning and to save time by reusing previous experiments. Additionally, the Cloudmesh App Store may function as a clearing-house of images, image templates, services offered and provisioning templates. Users may package complex deployment descriptions in an easy parameter/form-based interface and other users may be able to replicate the specified setup with.
Due to our advanced Cloudmesh Metrics framework we are in the position to further develop an integrated accounting framework allowing a usage cost model for users and management to identify the real impact of an experiment on resources. This will be useful to avoid overprovisioning and inefficient resource usage. The cost model will be based not only on number of core hours used, but also the capabilities of the resource, the time, and special support it takes to set up the experiment. We will expand upon the metrics framework of FutureGrid that allows measuring of VM and HPC usage and associate this with cost models. Benchmarks will be used to normalize the charge models.


\paragraph{Networking}


in{ceWe have a broad vision of resource integration in FutureGridources be with systems offering different levels of control from bare metal to use of a portion of a resource. Likewise, we must utilize networks offering various levels of control, from standard IP connectivity to completely configurable SDNs as novel cloud architectures will almost certainly leverage NaaS and SDN alongside system software and middleware. FutureGrid resources will make use of SDN using OpenFlow whenever possible and the same level of networking control will not be available in every location.


\paragraph{Monitoring}


To serve the purposes of CISE researchers, Cloudmesh must be able to access empirical data about the properties and performance of the underlying infrastructure beyond what is available from commercial cloud environments. To accommodate this requirement we have developed a uniform access interface to virtual machine monitoring information available for OpenStack, Eucalyptus, and Nimbus. In the future, we will be enhancing the access to historical user information. Right now they are exposed through predefined reports that we create on a regular basis. To achieve this we will also leverage the ongoing work while using the AMPQ protocol. Furthermore, Cloudmesh will provide access to common monitoring infrastructure as provided by Ganglia, Nagios, Inca, perfSonar, PAPI and others.

\subsection{Cloud Shifting}


We have already demonstrated via the RAIN tool in Cloudmesh that it is possible to easily shift resources between services. We are currently expanding upon this idea and developing more easy to use user interfaces that assist administrators and users through role and project based authentication to move resources from one service to another (see Figure \ref{F:shift}).


\begin{figure}[htb]
  \centering
    \includegraphics[width=1.0\columnwidth]{images/shift2.pdf}
  \caption{Shifting resources makes it possible to offer flexibility
    in the service distribution in case of over or underprovisioning.}\label{F:shift}
\end{figure}


\subsection{Graphical User Interface}


Despite the fact that Cloudmesh was originally a quite sophisticated command shell and command line tool, we have spend recently more time in exposing this functionality through a convenient Web interface. Some more popular views if this interface are depicted in Figure \ref{F:instances} hinting on how easy it is with a single button to create multiple VMs across a variety of IaaS. Also nice is that this not only includes resources at IU but also at external locations. Pushing this easy management in a more sophisticated experience for the user while associating one-click deployments that include the ability to deploy virtual clusters, Hadoop environments, and other more elaborate setups we provide an early prototype screenshot in Figure \ref{F:oneclick}.


\begin{figure}[htb]
  \centering
    \includegraphics[width=1.0\columnwidth]{images/rainbow.pdf}
  \caption{Monitoring the Service distribution of FutureGrid with Cloudmesh.}
\end{figure}


\begin{figure}[htb]
  \centering
    \includegraphics[width=1.0\columnwidth]{images/instances.pdf}
  \caption{Screenshot demonstrating how easy ot is to manage multible VMs accross various clouds.}\label{F:instances}
  \centering
    \includegraphics[width=1.0\columnwidth]{images/oneclick.pdf}
  \caption{One click deployment of platforms and sophisticated
    services that could even spawn multiple resources.}\label{F:oneclick}
\end{figure}

\section{Cloud Metrics}

Based on lessons learned from FutureGrid, knowledge of the resource
allocation and utilization is important to provide users and
administrators with a holistic view of the infrastructure in order to
guide better utilization overall and on an individual basis. This is
especially of interest in cloud deployments such as FutureGrid, which
supports more than 380 projects and 2300 users (as of April 2014). Due
to such a large user base resources could become over provisioned or
are not properly utilized by the users. Among the many services that
FutureGrid offers we have especially focussed on IaaS including
OpenStack, Eucalyptus, Nimbus, as well as batch systems to offer high
performance computing capabilities.  However, for this paper we will
restrict our discussion on the IaaS based monitoring components.
Other HPC related activities in regards to monitoring and metrics are
discussed in
\cite{ubmod,las12xdmod-kernel,las12xdmod-planing,las13xdmod,smith13info}

When FG initially started the existing IaaS frameworks such as
Eucalyptus, Nimbus, and OpenStack did not provide adequate support for
monitoring resource usage. Furthermore, a service with sufficient
monitoring capabilities across heterogeneous cloud IaaS frameworks was
not available. Hence, it was difficult to assess user utilization in a
holistic fashion. Additionally, we found that some IaaS frameworks
sucha as Nimbus, lack support for project allocations, a must have
feature to support project managed allocations as is the case in
almost every modern shared datacenter.  To overcome these missing
features and service offerings, we developed a {\em federated cloud metric
service} that aggregates the information from distributed clusters and a
variety of heterogeneous IaaS services, such as OpenStack, Eucalyptus,
and Nimbus. We name this service {\em Cloudmesh Metrics}.

The main components of {\em Cloudmesh Metrics} enable (a) the
measurement of the resource allocation across several IaaS platforms,
(b) the generation of data in regards to utilization, (c) the
comparison of data via definable metrics to mine the usage statistics,
(d) the display of the information through a convenient user
interface, (f) the availability of a simple command line interface and
shell language, and (e) the automatic creation of resource reports in
printed format for arbitrary time periods.

The services offered by Cloudmesh Metrics support requirements from a
variety of user communities. This includes individual users and users
part of projects (Section \ref{S:user-metric}), as well as administrative users
(Section \ref{S:resource-metric}).


\subsection{User- and Project-based Metrics Services}\label{S:user-metric} 

In order for users to use a variety of clouds it is important for them
to monitor and compare their resource utilization on them. In case the
usage is organized as a project the project related information needs
to be exposed while being able to clearly distinguish between
different projects. Furthermore we need to support an overall project
view.  Thus the requirement exists to present the data to the user
based on individual user utilization, group, utilization, or even
experiment utilization, where a particular experiment is analyzed
instead of just looking at the total utilization.

\subsection{Resource Provider-based Metrics Services} \label{S:resource-metric}.

For the resource provider it is important to have access to a holistic
view of the a variety of metrics across a the various clouds that
build the multi-cloud environment as part of a provider consortium. 
Summary information may be customized based on the requirements by
individual users, project leads, resource providers, site managers,
and funding agencies. This information is typically restricted to the
actual {\em resources} for which administrative access exists in order
to provide a holistic set of metrics.


\subsection{Metric Access for Multi-cloud Environments} 

Due to the different governance models between a private cloud managed
as part of a provider consortium, and the integration of resources
must be based on an access integration policy. In order to devise such
a policy, we need to be aware of the hierarchical access management
employed in clouds and depicted in Figure \ref{F:metric-hierarchy}.

\begin{figure}[htb]
  \centering
    \includegraphics[width=1.0\columnwidth]{images/metric-hierarchy.pdf}
  \caption{An example of a policy to estabilsh a access public clouds as part of 
    provider and user managed multi-clouds.}
  \label{F:metric-hierarchy}
\end{figure}

\subsubsection{Integration into a Provider Consortium Managed Multi-Cloud}

\newcommand{\ARROW}{$\Rightarrow$}

In the hierarchy we distinguish user \ARROW project with many users
\ARROW a resource serving many projects \ARROW a Provider having many
resources \ARROW and a Provider consortium with many providers. 

The provider consortium has a multitude of possibilities to extend
their resource offerings to its users while integrating public
clouds. This could include to replicate projects on public clouds, or
to assign a particular user access to an account that integrates
resources in a public cloud. In case Multiple clouds are offered this
integration could be replicated for them. Hence it would be possible
for a Provider consortium not only to provide private clouds as part
of a multi-cloud environment, but also public cloud offerings, allowing
access to these public cloud through a Provider Consortium managed
project or user on the public clouds. This is of especial interrest if
we try to gain financial advantages through volume discounts that
would otherwise not be available to the users. 
In Figure \ref{F:metric-hierarchy} we show one example for such an
integration policy where we simply map the existing projects and users
of the consortium to projects and users of a public cloud. A good
example where such integration is easy to accomplish is the HP Cloud
environment that uses OpenStack as it's IaaS framework. Due to
OpenStacks well documented interfaces it is possible to replicate the
user and project information and provide detailed charges to the users
ond projects in case the HP OpenStack cloud would be used. Naturally
it would be possible to devise other integration policies while for
example restricting access for just approved projects, or provide
access into different levels of the hierarchy.

\subsubsection{Integration into a User Managed Multi-Cloud}

Based on our experience with FutureGrid, we must also be aware that
users may have their own accounts and access to other clouds to
integrate them into a multi-cloud environment. The user may decide if
the metric information to such clouds are forwarded to the a
multi-cloud environment offered as part of a Provider
Consortium. However, in most cases the user will not share this
information. Hence the metric system ideally should be able to allow
users to integrate their own information into such a metric system in
order for the user to gain a more clear picture about their own
cloud usage not just in the consortium, but also in the public
clouds. We represent this clear division with in Figure
\ref{F:metric-hierarchy}. Explicit access policies must be defined to
allow the users or project to access the information provided by the
public clouds.


\subsection{Cloudmesh Metric Architecture}

The Cloudmesh metric architecture is based on the integration of a
authorized REST service, that utilizes a simple abstraction layer to interface
with the various cloud services to obtain needed information gathered
under authorization constraints. The data will be hosted in a NOSQL
database to allow mining of the data in map/reduce frameworks. Data
can be ingested either directly through the database via the API, or
through REST calls that are mitigated through message queues with
AMPQ. Adapters can be written to integrate new information providers
for other clouds. Policies can be used to limit the amount of
information presented to other users or projects.

\subsection{Cloudmesh Metric Service for FutureGrid}

To work towards the goal of a metric system for multi-cloud
environments, we start by basing our initial
development efforts on the extension of the cloud metrics services
that have been developed by FutureGrid for a multi-cloud environment
for resource providers within FutureGrid. Here we have limited the
services to a Provider Consortium that offers information of clouds
directly managed by the consortium. This includes OpenStack,
Eucalyptus, and Nimbus clouds on various resources. The information
access policy for using the resources is public as this is most
suitable to the goals of FutureGrid as a public testbed including
cloud frameworks. 

\subsubsection{Data Collector and Metric service}

%\begin{figure}[htb]
%  \centering
%    \includegraphics[width=1.0\columnwidth]{images/cloudmesh-metrics.pdf}
%  \caption{Design of Cloudmesh Metrics Architecture}
%  \label{F:metric-arch}
%\end{figure}

One of the base services needed is a data collector. It collects
relevant data from a variety of sources including resource databases,
logfiles, and data reporters. Hence, to integrate new cloud services
into the data collectors we have to define a data model, as well as
data sources that populate the data model with data. Currently data
collectors are available for OpenStack, Eucalyptus, and Nimbus but not
limited to these platforms. Dependent on the IaaS framework they
obtain the data from different sources such as log files or data bases
as listed in Table \ref{T:compare-iaas}. Information that is useful to
be collected include detailed information about the virtual machines,
the users and/or projects starting them, memory usage over the
lifetime of the VM, errors associated with vms during runtime, or ast
startup. One of the issues to be addressed is if such data should be
directly accessed in the production environment offered by the IaaS
framework. Practical experience with FutureGrid has shown that the
analysis of the data poses a significant amount of stress on the
originating resource making it impractical to offer a detailed report
and metric system on the original data sources. Hence it is important
that we replicate it when the information we request is involved in a
detailed analysis. For some smaler scale queries as the one posed by
users direct access is sufficient and desirable in case of a live view
of the system in order to provide information about how many VMs are
currently running, on which system and by whom.



%\newcommand{\YES}{\checkmark}
%\newcommand{\NO}{\textopenbullet}
\newcommand{\YES}{\ding{51}}
\newcommand{\NO}{\ding{55}}
%\newcommand{\YES}{$\oplus$}
%\newcommand{\NO}{$\ominus$}


\begin{table}[h!]
  \caption{Measurement of IaaS on FutureGrid}\label{T:compare-iaas}
  ~\\
  \begin{small}
  \begin{tabularx}{\columnwidth}{|l|X|X|X|}
  \hline
                 & {\bf Nimbus} & {\bf OpenStack} & {\bf Eucalyptus} \\
    \hline
    \hline
    \multicolumn{4}{|l|}{\bf Documentation of the Data Sources} \\
    \hline
       & \NO & \YES & \YES \\
    \hline
    \hline
    \multicolumn{4}{|l|}{\bf Data Sources} \\
    \hline
         & sqlite3 & MySQL & Log Files \\
    \hline
    \hline
    \multicolumn{4}{|l|}{\bf Metrics} \\
    \hline
    ~~vCPU core & \YES & \YES & \YES \\
    ~~memory & \YES & \YES & \YES \\
    ~~disk & \YES & \YES & \YES \\
    ~~instance type   & \NO & \YES & \YES \\
    ~~host & \YES & \YES & \YES \\
    \hline
    \hline
    \multicolumn{4}{|l|}{\bf Account Management Features} \\
    \hline
    ~~Users     & \YES & \YES & \YES \\
    ~~Projects & \NO & \YES & \YES \\
    \hline
    \hline
    \multicolumn{4}{|l|}{\bf Cluster} \\
    \hline
    ~~Alamo  & \YES & \YES & \NO \\
    ~~Foxtrot & \YES & \NO & \NO \\
    ~~Hotel    & \YES & \YES & \NO \\
    ~~India     & \NO  & \YES & \YES \\
    ~~Sierra    & \NO & \YES & \YES \\
    ~~Lima     & ?       &  ?      &  ?       \\   
    \hline
%    Region& \shortstack[l]{TACC$^1$, \\UF$^2$, \\UChicago$^3$, \\SDSC$^4$} & \shortstack[l]{TACC, \\IU$^5$, \\SDSC } & \shortstack[l]{IU, \\SDSC$^6$} \\
%    \hline
  \end{tabularx}\\
%  $^1$ Texas Advanced Computing Center \\
%  $^2$ University of Florida \\
%  $^3$ University of Chicago \\
%  $^4$ San Diego Supercomputing Center\\
%  $^5$ Indiana University\\
%  $^6$ in early 2014\\
\end{small}
\end{table}


\subsubsection{Metric Analyzer}

The data collected provides the opportunity to analyze it for specific
needs in a repeated fashion or provide filters and services for
further specialized analysis. The FutureGrid metric framework provides
therefore a metric analyzer component with a convenient interface for
analyzing the data not only on an automated fashion, but also
interactively through a simple metrics analyzer shell. Information of
interest include yearly, monthly, weekly, usage information by user,
project, resource, provider, and the agglomerated information. Our
scripting environment provides this information and is run at
predefined intervals or upon request. In future we will be enhancing
the service to allow users to schedule queries to conduct specific
analysis. To avoid repeated analysis, metric result caching is
conducted. Thus if a query has been executed in the past the result is
cached and returned without reanalysis (if not forced). To more easily
facilitate fast and distributed calculation of the results by multiple
users, we will base future versions of the Cloudmetric system on NoSQL
database technologies. 

\subsection{Metric Interface}

Early on we recognized the access the information and the metrics must
be provided through a variety of interfaces. This includes command
shells, programming API's, REST interfaces, graphical user interfaces,
and printable reports.

{\bf Interactive Command Shell.} To simplify the interactive use,
we have developed a python command shell called CMD3 that allows the
dynamic load of additional commands, thus making it ideal to define
new analytic methods on the fly if they are not provided by the
original toolkit.

{\bf REST API.} To allow easy access from Web frameworks, but
also integration form arbitrary programming languages, we are
currently building access through a convenient REST API.

{\bf Programming API.} We have provided a robust API interface
in python to access the basic  analytical functions useful for many
users and reused by the interactive command shell and the REST service.

{\bf Graphical Representation and Printable Reports.}

Using our basic API and command shell, we have integrated them into
the Python sphinx framework \cite{brandl2009sphinx} to expose the
metric data in a convenient form and present the data online via
charts ~\cite{highsoft2012highcharts} or in a PDF report. As sphinx
offers the export of data reports in PDF we leverage this framework
and do not have to develop a separate frame work for it. The sphinx
framework service is currently enhanced to allow customizable
interactive queries to the metric and data sources. The data can be
represented easily in various chart forms such as bar graphs, line
charts, or pie charts. A template for generating a quarterly and
yearly report of the data exists making adaptation to additional
resources or other provider consortia easy. Furthermore, the data can
be exported in a variety of formats such as JSON or csv making it
possible to use other tools such as excel for data post processing.
In Table~\ref{T:iaas-with-graph} we are including a limited number of
examples to demonstrate the various data representations of the
Cloudmetric system that are exposed to the users 


\begin{table}[h!]
  \caption{Metric visualization with graphs}\label{T:iaas-with-graph}
  ~\\
  \begin{center}
  \begin{small}
  \begin{tabular}{|p{0.2\columnwidth}|p{0.7\columnwidth}|}
  \hline
  {\bf Example} & {\bf Description} \\
    \hline
    \adjustbox{valign=t}{\includegraphics[width=0.2\columnwidth]{images/metric-line-chart.pdf}}
    &  
    Detailed display of virtual machine information including number
    of virtual machines, user count, memory utilization, disk
    utilization, project lead, etc. \\
    \hline
    \adjustbox{valign=t}{\includegraphics[width=0.2\columnwidth]{images/metric-histogram-chart.pdf}}
    & Summary information for periods to display aggregates of the
    metrics gathered by the system, such as number of vms by month for
    a user, project, or resource.  \\
    \hline
    \adjustbox{valign=t}{\includegraphics[width=0.2\columnwidth]{images/metric-pie-chart.pdf}}
    & Alternate representation of aggregated information in pie
    charts. \\
    \hline
    \adjustbox{valign=t}{\includegraphics[width=0.2\columnwidth]{images/metric-table-chart.pdf}}
    & 
    Alternate representation of agglomerated information in a table
    (exporatble as csv or json). \\
    \hline
  \end{tabular}\\
\end{small}
\end{center}
\end{table}

%%%%%%%%%%%%%%%%%%%%%%%%%%%%%%%%%%%%%%%%%%%%%%%%%%%%%%%%%%%%%%%%%%%%%%
\section{Conclusion}
%%%%%%%%%%%%%%%%%%%%%%%%%%%%%%%%%%%%%%%%%%%%%%%%%%%%%%%%%%%%%%%%%%%%%%

TBD
 
%%%%%%%%%%%%%%%%%%%%%%%%%%%%%%%%%%%%%%%%%%%%%%%%%%%%%%%%%%%%%%%%%%%%%% 
% Acknowledgment 
%%%%%%%%%%%%%%%%%%%%%%%%%%%%%%%%%%%%%%%%%%%%%%%%%%%%%%%%%%%%%%%%%%%%%% 
 
\section*{Acknowledgement} 
 
This material is based upon work supported in part by the National Science Foundation under Grant No. 0910812 and OCI 1025159. 

%\clearpage 
 
%\bibliographystyle{IEEEtranS}
\bibliographystyle{IEEEtran}
%\bibliographystyle{abbrv} 
\bibliography{% 
bib/vonLaszewski-jabref.bib,%
bib/cyberaide-cloud,%
bib/cyberaide-metric} 
 
\end{document} 






